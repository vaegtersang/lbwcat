\documentclass[../FGP.tex]{subfiles}
\renewcommand\thepage{\arabic{page}}
\begin{document}
%Tripartite division of the soul, because Triforce, and then a tripartite division of those:
% Mind: Reason, will, appetites 
% Body: Matter, form, an inexplicable vital principal (notably the completeness of Mind and Body makes the mind-body problem particularly thorny: both systems are complete of themselves and must be, so how do they link up?)
% Destiny: This subdivision is deeply muddled, but there is a sense that not only do all great world-historical personages appear, as it were, twice -- but they appear again and again in a cycle, and so do random shop keepers and ranchers and so on. These components separate out what parts do and don't recur and in what ways
% 
\newgeometry{textwidth=300pt,marginparwidth=110pt}
\part{World}
\section{A Begrudging Cosmology}
This is -- at least to me -- the ``eat your vegetables'' part of world-building. The parts of The Legend of Zelda involving ancient Goddesses and long-slumbering holy artefacts don't matter much to me. Unfortunately, the Goddesses and holy artefacts matter a great deal to the characters I am writing about, and besides one of them has an incorrigible scientific curiosity,%
  \marginnote{Which is funny, since I don't think there's really such a field of inquiry as ``science'' in this world. It might be better to say her curiosity is omnivorous and given systematic organization.}
which means that I need to have at least something of an understanding for how all of the myths and magic work in concert before she goes picking at it all.

\subsection{Whose Cosmos?}
  \begin{quote}
  \textit{Nor do I want it to be reputed presumption if a man from a low and mean state dares to discuss and give rules for the government of princes. For just as those who sketch landscapes place themselves down in the plain to consider the nature of mountains and high places and to consider the nature of low places put themselves high atop mountains, similarly, to know well the natures of peoples one needs to be [a] prince, and to know well the natures of princes one needs to be of the people.}

  \hfill--Niccolò Machiavelli\footnote{\label{note:prince}\textsc{Niccolò Machiavelli, The Prince} 4 (Harvey C.~Mansfield trans., Univ.~of Chi.~Press 2d ed.~1998)(1532).}
  \end{quote}
Continuing in my whining, most of this section -- in fact, much of this part -- is harried by the constant need to explain not only how the world is, but to itemize who believes what about the world. 
In most actual \emph{Zelda} games, the mythic past of Hyrule bears upon the story directly; Link sees the Triforce himself, hears first-hand from Zelda that she is the re-incarnation of Hylia, and so on. In my fic, the mythic past is most important for the way that it shapes my characteres understanding of their destinies and the expectations people have of them. Because these are questions of perception, and Hyrule is a world where different peoples know different portions of the legends, there will not be one set of expectations on a main character, but several. To take a concrete example, few in Skyloft know of Hylia-Zelda's incarnation, and in \emph{Ocarina of Time}%
  \footnote{\textsc{Nintendo, The Legend of Zelda: Ocarina of Time} (Nintendo 64, 1998)[hereinafter \textsc{Nintendo, OoT}]\label{note:oot}}
the story of the Triforce is explicitly kept secret to those outside the Royal Family and their Sheikah attendants. Meanwhile, in \emph{A Link Between Worlds}%
    \footnote{\textsc{Nintendo, LBW} \textit{supra} note \ref{note:lbw}}
and \emph{A Link to the Past}%
  \footnote{\textsc{Nintendo, The Legend of Zelda: A Link to the Past} (Super Nintendo Entertainment System, 1991)[hereinafter \textsc{Nintendo, LttP}\label{note:lttp}]}
the myth of the Triforce must be explained, suggesting that even if it is not secret then neither is it widely known. Such semi-secret pieces of lore might be called Royal Mysteries.\footnote{``Mystery'' here is in the sense of the original Greek {\Gr μυστήριον} which means not just a mystery in the English sense, but also a ``secret rite'' or ``a secret revealed by God, \emph{i.e.} religious or mystical truth,'' and in the sense of ``secret rite'' was used to describe mystery cults who worshiped one god or another, but kept their means of worship and some knowledge of the god secret to those outside the cult. \textit{\Gr Μυστήριον}, \textsc{H.G.~Liddell \& R.~Scott, A Greek-English Lexicon} (9th ed., 1996)({\Gr μυστήριον}, senses 1 and 4, with special citation to the cult of Demeter at Eleusis for secret rites in sense 1); \textit{see generally}, Kiki Karoglou, \textit{Mystery Cults in the Greek and Roman World}, \textsc{Metro.~Museum of Art: Heilbrunn Timeline of Art Hist.} (Oct.~2013), \url{https://www.metmuseum.org/toah/hd/myst/hd_myst.htm} (synoptic review of mystery cults in Ancient Greece and Rome).} If the tale of Hylia's incarnation is one such Royal Mystery a common person's understanding of what a princess ought to do will necessarily be motivated by different assumptions with different emphases than a royal tutor or Sheikah attendant would bring to the topic.%
  \footnote{\textit{See} \textsc{Nintendo, OoT}, \textit{supra} note \ref{note:oot} (the facts about Princess Zelda mentioned by the public are not that she holds the Triforce of Wisdom or is part of a perpetual chain of reincarnation -- the ones relevant to the game's plot -- but that she has prophetic dreams and is a tomboy).}
Insofar as Zelda is the incarnation of Hylia, the r\^ole Hylia had in guarding the Triforce is going to deeply shape expectations of the r\^ole a Hylian princess is expected to play in governing -- but only if the person with the expectations knows Zelda is Hylia's reincarnation. Commoners -- going off rumor and trivia -- may expect that much of a Princess' duty consists in sharing prophesies at events of religious or ceremonial importance. Meanwhile, those in the know of the Royal Mysteries would naturally focus far more on her duty to guard the Triforce of Wisdom. Elite and public expectations might come into \emph{partial} harmony over time even without public knowledge of the various mysteries the royal family keeps secret by the public's memory of what, traditionally, a princess has done in the past and expecting the current Princess to continue on this tradition, even if they do not have access to the underlying logic of why Zelda's duties are what they are.% 
  \footnote{\textit{Cf.}, \textsc{Niccolò Machiavelli, The Prince}, \textit{supra}, note \ref{note:prince}, at 6--7 (Translator's footnotes omitted.): \begin{quote}
  ``I say, then, that in hereditary states accustomed to the bloodline of their prince the difficulties in maintaining them are much less than in new states because it is enough not to depart from the order of his ancestors, and then to temporize in the face of accidents. In this way, if a prince is of ordinary industry, he will always maintain himself in his state unless there is an extraordinary and excessive force which deprives him of it; and should he be deprived of it, if any mishap whatever befalls the occupier, he reacquires it.''
  \end{quote}where hereditary princes not only benefit from, but rely upon, inertia and keeping the practices of their predecessors. Though certainly disputable in our own history, it is hard not to take the Legend of Zelda series as a whole as episodes vindicating this claim for the monarchy of Hyrule -- continuity is rewarded and the King and Princess, when overthrown, are quickly restored by good fortune and divine providence (\emph{i.e.}, the victory of Link), \textit{but see}, \textsc{Nintendo, WW}](there is a very long interregnum, in which all memory of the Hylian Royal Family is forgotten, and ultimately the land of Hyrule is left under the sea, with a deliberate decision for the heir to the throne to continue on as an explorer) even this counter-example, however comes with caveats, several generations on the monarchy has once again been restored, \textit{see} \textsc{Nintendo, ST}.} 
However, a look at actual history provides strong notes of caution against applying this harmonization hypothesis too eagerly.

The introduction of printing to northern Italy shows the diffusion of religious knowledge to the people by no means guarantees religious unity. Domenico Scandella, a Friulian miller born in 1532 and nicknamed Menocchio, learned to read and from his readings found ``the linguistic and conceptual tools to develop and express his view of the world[,]'' but even as his reading gave him ``an expository method based in the manner of the scholastics on the enunciation and refutation of erroneous opinions,'' he was as likely to draw on the opinions the Church set out to condemn as those it treated as doctrine.%
  \footnote{\textsc{Carlo Ginzburg, The Cheese and the Worms: The Cosmos of a Sixteenth Century Miller} 60--61 (John Tedeschi \& Anne Tedeschi transls., Johns Hopkins Univ.~Press 1980)(1976).}
Menocchio frequently went further than the claims he had read; he denied the divinity of Jesus,%
    \footnote{\textit{Id.} at 43 (``[I]t seemed a strange thing to me that a lord would allow himself to be taken in this way, and so I suspected that since he was crucified he was not God, but some prophet.'')}
the virginity of Mary,%
  \footnote{\textit{Id.} at 4 (denying virginity of Mary and denying Jesus' divinity \emph{again}).}
and the efficacy of almost all sacrements.%
  \footnote{\textit{Id.} at 10--11 (denying the efficacy of all sacrements but the eucharist, and then denying that the host is transformed into Christ's body to go seven for seven).}
Carlo Ginzburg -- whose historiography brought Menocchio to fame -- observes that, in the miller's reading, each text is refracted and filtered through already existing assumptions about the world and cosmos, and passages meant to prove or show one point may be taken as confirmation or inspiration in Menocchio's mind for another idea entirely.%
  \footnote{\textit{See Id.} at 34--36 (citing examples where details from books about Mary's life turn, in Menocchio's reading, into evidence that Jesus had a normal, not virginal, birth despite neither book actually claiming so).)}
Nor was this pattern of confirmation exclusive to religious reading. In fact, many of Menocchio's most formative readings were secular in character.%
  \footnote{\textit{Id.} at 41--50 (examining how stories in \emph{The Travels of Sir John Mandeville} and the \emph{Decameron} informed Menocchio's religious plurlism)} 
The availability of printed information, sanctioned or not, religious or not, had provided Menocchio ``to express the obscure, inarticulate vision of the world that fermented within him,''%
  \footnote{\textit{Id.}~at 59. Ginzburg also credits the Protestant Reformation with giving Menocchio the courage to express opinions that countered Church dogma. This may be inapposite to the Hyrule comparison, but it is also one of the points where Ginzburg is least convincing -- certainly the Reformation would have emboldened many people, and emboldened even Menocchio, but it is difficult to imagine that a person as strident and original in their beliefs as the Menocchio Ginzburg portrays needed even a grain of social permission.}
and if Menocchio had found a way to do this with the most unobjectionable odds and ends, a reasonable conclusion is that any printing culture will open interpretive problems more than it will allow a central authority to disseminate a canonical vision. Even more than early modern Italy, Hyrule has evidence of widely dispersed literary culture. Public signage is a widely-used way of communication, and \emph{A Link to the Past} and its successors have widely available books -- including the Book of Mudora which allows for the translation of Ancient Hylian and therefore direct reading of at least some of the mythic past in a way that Menocciho would envy. Thus, much of the Royal Mysteries are probably partly known, but also likely interpreted very differently by the populace. 

\subsection{Triforce}Though the Triforce is regarded as a single unit in most conversation,\footnote{Survey across games referring to its power as a single unit} it is frequently severed. This in fact comes in two varieties, the Triforce can be divided into three constituent pieces -- Power, Wisdom, Courage, one for each of the three Golden Goddesses;%
  \footnote{\textit{See}, \textit{e.g.}, \textsc{Nintendo, OoT} (if wished upon by one who does not carry all three virtues in balance, the Triforce will split, leaving only the piece the wielder most values -- the two other pieces will rest with those destined to hold them); \textsc{Nintendo, WW} (the Triforce appears when Link, Zelda, and Ganondorf meet in one place, uniting the three constituent pieces); \textsc{Nintendo, LBW} (the same); \textsc{Nintendo, SS} (Link must find and unite the three pieces of the Triforce hidden in Sky Keep)}
in other cases, these three pieces can be further subdivided.%
  \footnote{\textit{See}, \textit{e.g.}, \textsc{Nintendo, WW} (Eight shards of the Triforce of Courage are buried at the bottom of the great sea); \textsc{Nintendo, LoZ} (Zelda splits the Triforce of Wisdom, and scatters its pieces across dungeons of Hyrule to prevent it from falling into Ganon's hands).}
This suggests that whatever unity the Triforce might have, it is a unity that must be maintained by outside influence,%
  \footnote{Hylia, of course, is the most obvious candidate for this maintainer. Indeed, after she incarnates as a mortal, the Triforce seems to do nothing \emph{but} either sit, hermitcally sealed, in the sacred realm, or split into pieces.}
with the three Triforces of Power, Wisdom, and Courage proving to be distinct, perhaps even discordant, elements of a whole. 

\subsection{Hylia-Zelda}
The three Golden Goddesses created the Triforce, and then entrusted its care to the Goddess Hylia.\footnote{\textit{See}, \textsc{Nintendo, The Legend of Zelda: Skyward Sword} (Nintendo Wii, 2011)[hereinafter \textsc{Nintendo, SS}]\label{note:ss}(the Trifoce was ``handed down by the gods of old,'' and is ``protected by Her Grace, the Goddess.'')} Although the Triforce is an artifact of divine power, it cannot by used except by mortals, and therefore when she thought it necessary, Hylia incarnated herself so that she might make use of its power in resisting Demise.%
  \footnote{\textit{Id.} (``[Hylia incarnated herself], as you have likely guessed, so that the supreme power created by the old gods could one day be used. For while the supreme power of the Triforce was created by gods, all of its power can never be wielded by one.'')}
\subsection{The Cycle}

\section{Royal Duty, Royal Authority}\label{sec:world:royalduty} 
  \begin{quote}{\it
  ``Now if God, (as Bellarmine saith) hath taught us by Natural Instinct, signified to us by the Creation, and confirmed by his own Example, the Excellency of Monarchy, why should Bellarmine or We doubt, but that it is Natural?}''

  \hfill--Robert Filmer\footnote{\textsc{Robert Filmer, Patriarcha, or the Natural Power of Kings} (1680), \textit{reprinted in} \textsc{Filmer: Patriarcha and Other Writings} 1, 23 (Johann P. Sommerville ed., Cambridge Univ.~Press 1991). I should say I'm quoting Filmer with a good deal of irony here -- if ever a monarchy were unlikely to place special emphasis on patrilineal descent (as Filmer does, trying to tie all Kings back to Adam), it would be the monarchy where princesses in particular had special destines and powers.}
  \end{quote}

\subsection{Old and New} 

What does Zelda \emph{do} if being Princess is largely about letting the plan unfold?
To the extent that Zelda's job is prophesying, is it an analogue of other kinds of divination?
Is Irene a New or Hereditary Ruler in Machiavelli's typology?
This research detour started by trying to add depth to an idea of how Hylians understand the tension between doing good as a ruler and being a good person, or even making cold utilitarian trade offs. That question still needs an answer because, hilariously, Machiavelli only wrote about these in the context of new principalities because established ones had it easy. 
 ...Can we pull something from Aquinas' model in \textit{De Regno; ad Regem Cypri}? Maybe? 

Aquinas quickly recapitulates an argument originally from Aristotle that political communities form naturally to enable the full flourshing of human life. In the Aquinian version, because ``one man could not sufficiently provide for life, unassisted,'' humans naturally live in communities, and each specialize in some work.\footnote{%
  \textsc{Thomas Aquinas, On Kingship: To the King of Cyprus}, (Gerald B. Phelan, trans., I. Th. Eschman \& Joseph Kenny, eds.), \P\P4--6 \textit{available at} \url{https://isidore.co/aquinas/DeRegno.htm}. In the O.G.~argument, Aristotle traces the naturalness of cities by following a chain of progression: individuals by nature desire to reproduce, and so form families, villages are the natural outgrowth of a large family and form to provide goods one family alone could not obtain (such as milk, I think?), and cities are the natural outgrowth of villages, so cities -- which form to satisfy the most elevated of human needs -- are likewise natural. \textit{See,} \textsc{Aristotle, Politics} 2--5 (C.D.C.~Reeve, trans., 1998)(Roughly Bekker lines 1252a 24--1253b) Aristotle, however, does not directly acknolwedge the need for specialization in the work of each individual though it is implicit in the claim that the goods provided by a household are different than those provided by a village.} % {\Gr Εὐνομία} {\Gr Εἰρήνη}}.
Test.



\section{What Does a Witch Do Exactly?}

\subsection{Witchcraft}
\subsubsection{Astrology: How is it Supposed to Work?}
You've put some deeply paradoxical constraints into it from your own writing: Irene doesn't respect it because it's all math and life is never all math. But at the same time, there's a set piece here: they're sitting beneath the stars and he's reading off what's behind the clouds covered over. New observations provide new data, which suggests this isn't a fully deterministic system, but we need it to be highly deterministic for Irene's objection to make sense. 

This sucks, but here is the best I've got as a synthesis: Stars influence people by some elaborate celestial pull,\marginnote{I think this woo is ``stars and a component of the soul are made of the same stuff, so the soul moves to find an equilibrium between starry pulls, and the soul in repositioning itself guides motivations and behavior''} this is ongoing (i.e., it's not all set at time of birth) but given the predictability of celestial motion, if you have a birth chart you can iterate out and see what influences will be tomorrow, and the next day, and the next, and so on. Weather patterns -- clear skies, cloudy skies, high refraction, low refraction -- filter and can alter the influence of stars, but this is a very small term next to the overall position of stars.  
\end{document}