\section{Irene} 
  \subsection{Problems/Motivations}
  \subsubsection{Doomed by The Narrative}\label{sec:DoomedByTheNarrative}
  Irene is -- obviously, canonically -- ambitious. She's ``the best witch of [her] generation''\footnote{\label{note:lbw}\textsc{LBW} at .} and wants to develop that talent and gain recognition. Irene is very happy with her rapid graduation from apprenticeship, and the quick growth of her own business as a witch; or she would be under other circumstances, but despite what should be progress, Irene knows that whatever recognition she gets as a witch will be overshadowed out by her r\^ole as a Sage in the Great Cycle that is Hyrules driving myth. She likewise is very happy with her relationship with Link, but is reluctant to name it for similar reasons (if they dated, there is a whole trope about sages in love with the Hero, and there too, the thought of being reduced to The Narrative is unbearable for her -- on top of that, she tacitly assumes he's destined to be with Zelda, so why dwell on it). The concise version of this is Irene knows what she wants, knows what she doesn't want, and knows that getting one will inevitably lead to the other, so what now?
  
  \paragraph{Attitude toward Destiny} Irene's primary problems with being a Sage are concrete -- she wants to make her own name, and being a Sage has already given her one that's hard to shake -- but finding out she is a Sage is also spiritually disorienting. She never doubted the Goddesses\footnote{Technically, Irene did not expect the Goddesses Themselves would have a plan for her, the Three aren't all-seeing the way the way the God of Abraham is, but They do have various agents who handle the lives of normal people, and Irene is sure they have a destiny for her.} had a plan for her, but she expected to find that plan out experimentally. The Goddesses shape events, but that isn't reason not to decide what you want to happen or where you want to go and start moving. If the Goddesses intend for you to get there you will, and if they have something different planned you won't. Simple as. Say your prayers, consult the fortune tellers, and keep working. And this worked fine when she thought that she would reach her destiny -- best witch in the world, preferably -- after a lifetime of hard work. Once her destinty seems to amount to ``sit around and be famous,'' and she doesn't have anything to \emph{do},\marginnote{I don't know if Irene realizes it but she wants destiny as something to pursue, not something to have. Would being a world-famous witch actually make Irene happy? Maybe, but I'm not sure of it.} that attitude toward destinty falls apart and she isn't sure how to think about destiny or the Goddess' plans instead.
  
  \paragraph{Link} And then, apart from her own mess, she's also concerned for Link, whose obligations as a big hero are running him ragged. Very few of these are actually dangerous, his work is mostly representing the crown, mediating disputes in Hylian villages, and making diplomatic appearances around Hyrule and neighboring lands, but though there's little chance for him to get hurt, he returns from riding circuit consistently exhausted and cranky. Irene understands why Link makes all these appearances, and she guesses it's better than fighting monsters, but it truly seems too hard on his mental health, and there's something that feels distinctly unfair to her that being so good at adventuring lands him a no-adventuring job he hates. (And then, a tiny selfish part of Irene doesn't like how frquently Royal duty takes Link away.)

  \subsubsection{Meddling} The problems above are terrible, and trying normal things has gotten her nowhere. She's talked with the Fortune Teller about the whole Sage and Destiny thing, and he doesn't see any way for her to get out from under her reputation. She's talked to Zelda about how she deals with having what everyone thinks of her decided in advance, and Zelda is broadly sympathetic, but not very understanding. Only partly related, she tries also nudging Zelda toward not sending Link out so often, but every trip Link makes does so much good for Hyrule and its people, and it's not risking his life, that Zelda simply doesn't know how she could \emph{not} ask it all of Link, even if it sometimes burdens him. All of this is bad news, but the refusal to ease up on Link infuriates Irene.

  All she has left are drastic measures. Witches are allowed to help people with problems when asked, but have strict professional rules against meddling and even stronger rules against casting powerful spells without other witches agreeing it's a good idea; Irene is about to break all of those. Irene feels tremendously guilty, and more than a little bit ashamed because she takes pride in being a good witch and rules of professional conduct mean something to her. She's also very anxious about what will happen if her Gram finds out, but the current situation is intolerable and she has no other way to change it. 
  
  So she casts a powerful spell and asks the Goddesses for her not to be overshadowed, for Link to get a chance to rest, and -- because she can't stop stewing on it -- for Zelda to have a better sense of what she's putting Link through.

  \subsubsection{Swapped}She wakes up the next day and finds she now is Witch-Crown Princess of Hyrule. She does not find out until later that Link has been turned into a cat, and it is much later still that she discovers Zelda is now out in the woods of Hyrule fighting monsters as the chosen hero. Irene has solved her problems with Zelda and Link but not in ways that she would ever have wanted, has not actually solved any of her own problems at all, and has also added exciting new problems to her situation, including:    
  
  \paragraph{Crown Princess of Hyrule (Personal)}\label{sec:witchprincess} Being Witch-Crown Prin\-cess is terrible! She's not overshadowed anymore, which is technically what she wished for, but she still is stuck with a r\^ole and perceptions of who she is that she has no control in shaping. Worse, being a Sage blocked her own ambitious but didn't intrude into her personal life. Being a princess, however, comes with tremendous amounts of intrusion by definition. I don't know how far down this road I want to go, exactly. Is it just limits on personal expression in the form of royal courtesies that grate upon Irene? Or does it go all the way to arranged marriages and expectations that Irene has a retinue that follows her everywhere?\footnote{For real historical counterparts to this intrusive retinue, and an example of how bad this could get, \emph{see, e.g.}, \textit{Groom of the Stool}, \textsc{Wikipedia}, \url{https://en.wikipedia.org/wiki/Groom_of_the_Stool} (last visited Dec.~30, 2023).}

  But even if I don't know how far I want to push this angle, or how I want to resolve it in the end, I know why I want to. For one, there's a lot of good tension to wring from this; Irene seems like the kind of person who needs a decent amount of time to herself to study, which she won't get as Princess, so there's her first problem. Irene really did not have marriage -- let alone political marraige -- on her radar, and though that probably won't be something that happens for several years, it also is a big looming problem.\marginnote{I'll fess up to wanting to purusue this angle at least in part because a scene where Irene meets some foreign Prince Suitor she might be arranged to marry in five years and is not remotely interested, and also has a cat angrily pacing figure eights around her and brushing up against her legs as the poor Prince Suitor tries to introduce himself strikes me as potentially very funny.} Irene doesn't know how to do the public-facing parts of her job as princess\footnote{Discussed \textit{infra}, section \ref{sec:machiavelli}.} but she also has no idea how to be a witch while also being an authority figure. Witches give good advice, and sometimes perform helpful spells, but they're supposed to live outside of formal governance, not at the top of it. If she gives advice as a princess, it comes out as orders no matter what she does. Everything about this, in a way, feels like a real mokey's paw wish. Irene gets to interrogate the connection between having renown and having what she really wants, and struggle with ``did the Goddesses just do this to reprimand me?'' Secondly, it's an interesting flip to some of Irene and Link's secondary problems; Link has all the time in the world (because he's a cat) and Irene is booked with being both a witch and covering royal duties. Thirdly, although Irene's complaints in Section \ref{sec:DoomedByTheNarrative} are all fair, there's enormous irony that she complains about them to Zelda who -- though she is at peace with it -- has had a far more rigid r\^ole in myth foisted upon her since birth, and choosing Irene to be the point-of-view character for most of the story sets Zelda up for unsympathetic treatment in the story. Leaning into the personal challenges of being a princess is nice for balancing this by getting some of Zelda's experiences as Princess on the page, even if Zelda doesn't actually show up in the story again until much later. 

  \paragraph{Crown Princess of Hyrule (Official)}\label{sec:machiavelli}
      \begin{quote}
      \begin{verse}\begingroup\it 
      The hand rules pity as a hand rules heaven;\\
      Hands have no tears to flow.\endgroup\\
      \hfill -- Dylan Thomas\footnote{Dylan Thomas, \textit{Three Poems}, \textsc{New Verse}, Dec.~1935, at 15, 16, \textit{available at} \url{http://www.modernistmagazines.com/media/pdf/268.pdf}. As will be discussed in a moment, much of Irene's ruling is done via adjudication, and is framed as the pain of picking a winner and a loser out of two claimants, both sympathetic. This is, in one way, much of what adjudication is, but on a second pass, it would be well to think carefully about whether any judgement can be separated from more straightforward violence, and whether there's something worse afoot than merely having to suppress sympathies for claimaints who may not get what they want. \textit{See},  Robert Cover, \textit{The Supreme Court, 1982 Term--Foreword: Nomos and Narrative}, 97 \textsc{Harv. L. Rev.} 4, 40-44 (1983); \textit{see generally}, Robert Cover \textit{Violence and the Word} \textsc{95 Yale L.J. 1601} (1986).}
      \end{verse}
      \end{quote}
    Because witches practice at least some medicine I think Irene is acquainted with the idea that she cannot help every one, though not yet \emph{well}-acquainted, but nevertheless cold utilitarian logic is offensive to her. She has thus far in her life and work been given one charge at a time, help this one family with their crops, cure this one broken arm, brew a new potion to keep the blacksmith cool at his work on summer days. None of these conflict, and though she may not succeed, she has never been made to choose whom of two incompatible people to help, or whom to help at the obvious expense of another.\footnote{%
      At least not directly, but as Charles Fried observes of doctors and lawyers, ``[t]he professional ideal authorizes a care for the client and the patient which, exceeds what the efficient distribution of a scarce social resource (the professional's time) would dictate.'' Charles Fried, \textit{The Lawyer as Friend}, \textsc{85 Yale L.~J.} 1060, 1061--62 (1976), \textit{available at} \url{https://dash.harvard.edu/handle/1/23903316}. Fried distinguishes lawyers from doctors in that, though both can make impersonal decisions to give care to one person and indirectly disadvantage a hypothetical other person who does not receive care, lawyers can in some cases go further and -- because of the antagonistic nature of the legal process -- use ``tactics which procure advantages for the client at the direct expense of some identified opposing party.'' These tactics may not even be things which we would, outside the context of being a lawyer, count as ethical. \textit{Id.}, 1062.  Irene's r\^ole as Crown Princess sits in an interesting middle where the person whom she is disadvantaging is not hypothetical but is present for the injuy, but -- unlike in the lawyer case -- there's no reason to think she is stepping outside the bounds of ethics to do it.} 
    This, however, is much of a Crown Princess' job: as Hylia guarded and kept unified the three discordant pieces of the Triforce, Irene is now expected to keep unified and happy all the people of Hyrule.\footnote{\textit{See infra} section \ref{sec:world:royalduty}.} This extends not just to difficulties of how to set policy, but to adjudicating disputes that anyone, high or low, has a right to bring to Her Highness. Judging in these cases is doubly hard on Irene, both because she must disappoint -- and sometimes substantially harm -- someone in person, and also because her duty to adjudicate sits uneasily with her work as a witch. As a witch she is advisor, but it is entirely possible now that a person may, after acting on her advice, come before her as a disputant, and she must judge impartially when she has already involved herslf in the dispute. This limits her freedom to speak as an advisor, for fear that the advisee may come back as a disputant claiming they were doing only as she told, and tangles her neutrality as judge, because any judgment for someone she has previously advised might give the appearance of favoritism. 
  
  \subsection{Character Relations}
  \subsubsection{Link}
  \paragraph{Canon}
  In the game, Irene's relation to Link has three identifiable stages. Before being captured by Yuga, Irene's attitude is distinctively transactional, and the aide she offers is given begrudgingly, and with an expected payout at some future time always in view. Even here, however, there is a soupçon of compassion beneath.\footnote{\label{note:nicegruff}\textit{See e.g.}, her comments of ``I hope you appreciate this. You know I’ve got a life of my own, right? Things to do? Places to fly? But it looks like you’re having a rough slog, so no problem.'' and ``Are you REALLY the green thing I’m supposed to be taking care of? Cuz you look like you’re doing just fine. Well, whatever. Nice to see you now and then,'' in \textsc{Nintendo, LBW}, \textit{supra} note \ref{note:lbw}.}  After her rescue at the end of the Desert Palace, she gives a brief speech which is, honestly, in several ways discordant from her characterization before and after. However, it is the longest piece of familiar speech during the game:\begin{quote}
  H-hey! You came to rescue me? Well, I\ldots I don’t know what to say, Link! But, uh, HEY! My fortune came true after all! You took your sweet time getting here, didn’t you?! I was in big trouble! And I’ve got to get back to my gram! That’s all right. I forgive you. Just don’t take that long to rescue me next time. And don’t make any of the other Sages wait that long either. C’mon get to saving the rest!\footnote{\textit{Id.} If spoken to again she adds ``What\ldots{}? Want more thanks? Tell you what, Link. Save all seven of us Sages, then I’ll write a big, long thank-you letter. But for now\ldots{}? Just be careful, OK?''}
  \end{quote}
  I struggle to read the ``you took your sweet time getting here'' portion as anything but mismatched \emph{tsundere}-ism,\marginnote{She is very obviously ambitious and open about being busy before and after, but I don't really feel like ``what took you so long'' is at all of a piece with ``I have things to do.'' Is this just me?} which she doesn't really have before or after this one speech, so I'm ignoring it. However, the casualness reference to the ``next time'' he rescues her contains the germ of what I think is a major strand of their relationship going forward: an assumed partnership where Irene will always have his back, and he'll always have hers. The transactionality from before is not gone, but is is transmuted into a warm, teasing exchange.For example, ``You’re really wearing out my poor li’l broom. You’d better buy me a new one once I’m out of here. Otherwise, uh\ldots{} how will I give you more rides?''\footnote{\textit{Id.}} Irene's ability to provide useful transport remains a source of both personal pride as a witch and also something she can do for her friends enthusiastically.\footnote{\textit{Cf.}, ``Hey! Have you met Mother Maiamai yet? I hope so. I’ve heard that if you help her, she’ll do nice things for you. I’m a li’l jealous how much she can help you out. She’s got POWER. My broom will have to do. Off you go!'' \textit{Id.} (Her broom can partly substitute for having other powers, insofar as they are both helpful, but she is envious of magic that is more dramatically helpful).}
  
  After she's rescued from the painting, Irene also is more openly caring about Link's well-being, frequentlty inquiring after his health and warning him of dangers.\footnote{Examples of this open concern include: ``Aren’t you tired? Look, I’m delighted to give you a lift, but don’t forget to rest sometimes,'' and ``You look like you’re used to battle. But don’t let your guard down. And, hey, here’s a tip about bottles. You know you can keep more than just potions and fairies in them? Yeah, apples and hearts too. So do that. It’ll keep you from keeling over. No one wants to see you get hurt.'' \textit{Id.}} I don't think this concern develops \emph{ex nihilo} once she's rescued Link, rather I think the generalized humanitarian compassion she expresses under layers of impatience when they first meet becomes less masked, and -- once situated in the context of a partnership -- that generalized compassion transmutes into close attachment. Despite that closeness, much of the partnership occurs at arm's length. Irene is there to bail him out of danger should things ever turn bad, but he largely doesn't discuss what monsters he has encountered or dungeons he has explored unless a point of magic comes up on which he needs her advice. This isn't necessarily done out of any coldness on his part, but the quest is work and work is always a boring subject of conversation -- even when it looks like an adventure to someone on the outside. Irene understands this, but even so wishes he were more conversational about ``work,'' and cannot help but often wonder what he's up to between times they meet in much the same way it's impossible to have an outdoor cat and not wonder what they're doing off in the woods.\footnote{Recent research suggests outdoor cats do not travel that far from home. \textit{See, e.g.}, R.~Kays \textit{et al.}, \textit{The small home ranges and large local ecological impacts of pet cats}, 23 \textsc{Animal Conservation} 516, 519 (2020)(mean cat ``home range'' was 3.6 hectares, and heavily concentrated in disturbed -- as opposed to wild -- areas); Helene Ane Jensen \textit{et al.}, \textit{Movement Patterns of Roaming Companion Cats in Denmark--A Study Based on GPS Tracking} 12 \textsc{Animals} 1748, 1755--56 (2022) (median cat ``home range'' was 5 hectares, with caveat that inter-cat variation was enormous); Martina Cecchetti \textit{et al.}, \textit{Spatial behavior of domestic cats and the effects of outdoor access restrictions and interventions to reduce predation of wildlife}, \textsc{Conservation Sci. \& Prac.} February 2022, at e597, \url{https://conbio.onlinelibrary.wiley.com/doi/epdf/10.1111/csp2.597} (median ``home range'' of cat is 1.5 hectares, though cats with unrestricted outdoor access have larger home ranges). \textit{See also} Krysten R.~Vitale, \textit{The Social Lives of Free-Ranging Cats} 12 \textsc{Animals} 126 (2022), \textit{available at} \url{https://www.ncbi.nlm.nih.gov/pmc/articles/PMC8749887/} (Literature review of free-range cat social behavior). Irene knows this, but also knows that if Link were a cat (foreshadowing!) he would be out on the long tail of cats that travel widely.}  

  \paragraph{Pre-Fic}They also just hang out in their preciously scarce free time. Easy moments fishing, appreciating natural beauty and tranquility in the quiet of the sacred grove or under the bridge of western Hyrule, talking idly and relishing the twofold miracle that they have time together and time free of obligations. If they had their wishes, there would be much more time like this, but they both keep busy schedules and so this is a favorite but atypical interaction for them. Instead, much of their friendship continues at arm's length or is mediated through some kind of work. Link comes by Gram's cottage and helps do potion prep and they chat. When he comes back from adventures in foreign lands he always has bottles and bottles of novel monster parts, and many of them go to Gram but Irene gets first dibs on the gift. If he gets a new magical curio, he shares that too. Often their schedules don't match and so he drops them at the doorstep to the cottage. She still lends him rides within Hyrule, of course, and is looking for ways that she can get her broom to go further. Training with Gram has thus-far prevented her from ever travelling with him, but she sends him off with protective charms and newly concoted potions whenever she can. She doesn't know it, but he deliberately talks her up when travelling abroad too to hasten that Famous Witch thing she's after. If she did know I think she would regard it as both amazingly touching and also a problem because of The Narrative.\footnote{\textit{See} \textit{supra}, section \ref{sec:DoomedByTheNarrative}.}

  \subsubsection{Gram}
  \paragraph{Canon}\label{gram:canon} When isolated in the Sacred Realm, Irene frequently alludes to missing her grandmother,\footnote{\textit{See} \textsc{Nintendo}, \textit{supra} note \ref{note:nicegruff} (``You know, I really miss my gram. Can’t wait to see her again'' and ``Uh, you know what my gram says? It’s good to walk. Stop to smell the roses. Pick up monster parts. I wonder how she’s doing\ldots{} Miss my gram'').}
  and when given freedom to roam frequently runs errands for her.\footnote{\textit{Id.} (``Normally I don’t take passengers, but I’d rather haul you all over Hyrule than face disaster. Anyway, gotta fly. I have errands to run for my gram. Later!'' and ``Yeah, yeah. I heard you. Clang, clang. I was busy helping my gram!'').} Her grandmother mediates much of her relationship with Irene through work, using potions to regulate the amount of attention directed toward her.\footnote{Before Irene is captured, her grandmother tells Link ``I keep myself busy with my potions, or else I find myself fussing over my granddaughter Irene too much.'' \textit{Id.} After Irene's capture, the Witch keeps herself busy in her work to avoid worrying. \textit{Id.} (``I know, I know\ldots{} This isn’t the time to be mixing potions like nothing’s wrong\ldots But I just can’t calm down if I’m not keeping busy\ldots'')} The Witch also admires her granddaughter's resilience.\footnote{\textit{Id.} (``That dear girl. No matter what trouble she gets herself into, she always bounces back with a smirk on her face.'')} Transactionality pervades conversations with the Witch,\footnote{Even her expressions of gratitude at Irene being rescued are transactionally-focused: ``My granddaughter and I will be forever indebted to you.'' \textit{Id.}} and though canon gives little explicitly noted, it is easy to imagine being the Witches grand-daughter as an extensive series of errands, mentoring, and small favors. There's plenty of love, but it is always expressed in the medium of some practical interaction.\marginnote{\emph{I.e.}, there's abundant love, but very little done ``just because.'' Gifts on special occaisons like birthdays and whatever they have in lieu of Christmas, I suppose. Otherwise, it's all mixed with work.} And though Irene returns the favors gladly because she loves her Gram back, I think this has led her to be extremely bad at openly expressing feelings.

  \paragraph{Pre-Fic}
  Parts of this section I think prob should just be what taking on more Witch duties from Gram entails. I think that probably is she's trusted to do lab prep (brew common mixtures that are pre-reqs to more elaborate stuff, get the staples -- red potion whatever -- going) and has taken up divination as well from the fortune teller (but cannot talk about that with her gram)
%  \footnote{}
%  \footnote{``Oh, heavens! My dear grandchild Irene was snatched! It happened so fast! Some awful man came through, very full of himself and transformed her into\ldots{} a painting! I-I-I was flummoxed! I just stood there helpless! Oh, my dear granddaughter-GONE! And she’d just been saying how she’d made a new friend of late\ldots{} I know, I know\ldots{} This isn’t the time to be mixing potions like nothing’s wrong\ldots But I just can’t calm down if I’m not keeping busy\ldots''}
  %\footnote{``Heh hee hee! Have you something to tell me? What! You saved my Irene? Wonder of wonders, is it possible? That dear girl. No matter what trouble she gets herself into, she always bounces back with a smirk on her face. My granddaughter and I will be forever indebted to you. Speaking of bouncing back, don’t forget about my potions. I’d hate for you to fall in battle out there!''}
  \subsection{Character Questions}
  \begin{enumerate}
    \item \textit{What’s the lie your character says most often?}\footnote{Taken from judasrpc, \textit{Weirdly Specific but Helpful Character-Building Questions}, \textsc{Tumblr} (Jun.~23, 2022), \url{https://www.tumblr.com/judasrpc/687874557884907520/weirdly-specific-but-helpful-character-building}.}\label{irene:characterqs:lies}
      If we're going by lies that she tells herself as well as everyone else, the answer is that she has a clear ambition in life: to be recognized as the greatest witch in the world. In fact, she wants work that she values, and can do well, but the desire is much more about process than it is the result. She wants a rewarding job she can keep doing. If we can only count lies she knows are lies, I think she spends a lot of time in conversation concealing her fear of foreign travel. She'll do it someday, she kind of needs to if she's to be a world-famous witch, but inside she's happy putting it off to help her Gram.  
    \item \textit{How loosely or strictly do they use the word ‘friend?’}
      Everyone she has met once and had a pleasant interaction with is a friend. Every one of her and Gram's customers. Every one she meets while foraging for potion ingredients. She has hundreds of friends. She just, frequently, doesn't get to meet them again. I think that broad read of friend is the product of growing up very, very isolated out in the wilds of Zora's river so whoever she met on rare occasion was immediately a treasured contact. 
    \item \textit{How often do they show their genuine emotions to others versus just the audience knowing?}
      Given her druthers, she'd show all the emotions she knows how to show quite openly.\footnote{The qualification ``she knows how to show'' is quite important. See questions \ref{irene:characterq:emotionallyconstipated}, \ref{irene:characterq:iloveyou}, and \ref{irene:characterq:toughlove} \textit{infra} at \pageref{irene:characterq:emotionallyconstipated}.} However, circumstances -- the need to present a professional case and avoid The Narrative, the need to comport herself Royally, have her quite reserved in most of this story and I think that compounds the burdens she's carrying quite severly.
    \item \textit{What’s a hobby they used to have that they miss?}
      Natural history; finding weird frogs and bugs in stream beds and under leaf litter. She used to get some in while foraging for potion ingredients, but demands on her time are such that when she's out in nature now there's rarely time to just explore and see what's around her except for finding things relevant to what she's brewing.
    \item \label{irene:characterq:emotionallyconstipated}\textit{Can they cry on command? If so, what do they think about to make it happen?}
      On command? I'm not sure she can cry at all. This poor girl is, in areas of affection and sadness, the most emotionally constipated character I have ever written.
    \item \textit{What’s their favorite [insert anything] that they’ve never recommended to anyone before?}
      She doesn't have anyone to share them with -- but Irene has extremely detailed opinions about which air routes on broomstick are pleasant and which are kind of a slog. Her very favorite is an S-bend from Kakariko down to the lake. About a thousand feet up you get a warm wind almost any time of day and it's just the most pleasant, grounding thing.
    \item \textit{What would you (mun) yell in the middle of a crowd to find them? What would their best friend and/or romantic partner yell?}
      I feel like this is the kind of question that only works if you're already close to the character? If I don't know Irene well I'm just shouting her name in the crowd to find her. In \emph{theory} you could yell something unflattering about divination and fortune telling and bait her out of the crowd, but the two people close enough to try are just not the baiting kind (her Gram) or too practical (Link, who has that bell, which is even better as bait).
    \item \textit{How loose is their use of the phrase ‘I love you’?}\label{irene:characterq:iloveyou}
      She has never used it.
    \item \textit{Do they give tough love or gentle love most often? Which do they prefer to receive?}\label{irene:characterq:toughlove}
      I don't think she's particularly tough, but gentle doesn't sound right either. I think that she primarily expresses \emph{care},\footnote{See \ref{gram:canon}, \textit{supra} at \pageref{gram:canon} for a clearer definition of the distinction.} I don't think she responds well to tough love at all -- things of the form ``you need to do this!'' trigger a fairly instinctive ``what's your problem? Why are you directing me?'' except for Gram, who in a lot of ways isn't that direct. Receiving gentle love is probably welcome, but the kind of thing that she isn't fully sure what to do with. 
    \item \textit{What fact do they excitedly tell everyone about at every opportunity?}
      All about palmistry! She's gone from not just being impressed by the fortune teller to picking up a lot of fortune telling herself and thinking it is such a cool complement to witchcraft and being a witch who has other specialities is just so exicting to her.
    \item \textit{If someone was impersonating them, what would friends/family ask or do to tell the difference?}
      Irene keeps most of her potion ingredients in a set of cubbies. To an outside observer it looks like they are stored more or less by putting ingredients wherever a spot is free. In actuality, there's a highly unique, comprehensively defined, order to keep like ingredients apart (for fear of them interacting in the cubby). To spot a fraud Link or Gram would just have to leave out some monster tails and wait to see the purported Irene shelve them.
    \item \textit{What’s something that makes them laugh every single time? Be specific!}
      The way injured octorocs deflate. Goofiest thing in the world. If Hyrule had Cape Rain frogs in distress,\footnote{\textit{See}, \textit{e.g.}, About Nature, \textit{This Frog Screams when it Gets Scared on Something} \textsc{YouTube} (Dec.~20, 2016), \url{https://www.youtube.com/watch?v=ZHnEiPGxsB8}.} she'd die laughing at those for analogous reasons.
    \item \textit{When do they fake a smile? How often?}
      Sometimes with client work she'll have to act more cheerful than she is. Kakariko is weird and seems especially to expect it when she does business out there. Thankfully that's not more than once a month or so and usually she's in a good mood when working anyway.
    \item \textit{How do they put out a candle?}
      She points at it with three fingers, wiggles them, and wills it out.
    \item \textit{What’s the most obvious difference between their behavior at home, at work, at school, with friends, and when they’re alone?}
      %TODO
    \item \textit{What kinds of people do they have arguments with in their head?}
      Her Gram, mostly, who is part regular conscience, and very much her professional conscience as a witch. She'll make up dialogues sometimes with people who she's frequently advised as a witch, too, to suss out what they're angling for as clients.
    \item \textit{What do they notice first in the mirror versus what most people first notice looking at them?}
      For her? Eye check first thing -- one sad fact of flying is you're often getting bugs caught in your lashes. For others? Witch costume (which is partly so eccentric to distract from the bugs that might be in her lashes).
    \item \textit{Who do they love truly, 100\% unconditionally (if anyone)?}
      Same two names you're going to be seeing this entire section: Link and Gram.
    \item \textit{What would they do if stuck in a room with the person they’ve been avoiding?}
      Wait. How stuck? Are the exits actually sealed? Confinement is 100\% her biggest fear and she will team up with anyone -- no matter how much she had been avoiding them -- if it means escape. Metaphorically stuck? She'll try to fake pondering a crystal ball rather than talk and make eye contact.
    \item \textit{Who do they like as a person but hate their work? Vice versa, whose work do they like but don’t like the person?}
      I think Irene has vexed opinions about a lot of merchants in Kakariko. Talon at the Milk Bar is very friendly but so unfocused and can't reliably stock anything. The carpet merchant\marginnote{Not actually in \textsc{LBW} but adding him in for this story.} is a great citizen -- helpful to all his neighbors -- but he sells at such a markup he's useless. 
    \item \textit{What common etiquette do they disagree with? Do they still follow it?}
      If there aren't caustic substances left out, feet go \emph{on} the table/desk. Gram has tried to scold her out of it, but no luck. I think in very official settings she'd be capable of holding back a little bit, but would probably still do it in private no matter how fancy the desk/table.
    \item \textit{What simple activity that most people do / can do scares your character?}
      Honestly I think Irene bats 100 here. There are things she finds offputting or distateful,\footnote{\textit{See} question \ref{irene:characterq:workhates}, \textit{infra}, at \pageref{irene:characterq:workhates}} but nothing that scares her.
    \item \textit{What do they feel guilty for that the other person(s) doesn’t/don’t even remember?}
      As a young witch there were a lot of potion brewing attempts that blew up in the face of quite a few neighboring Zora. She remembers all of them, and pretty much all the Zora washed it off and never thought of it again.
    \item \textit{Did they take a cookie from the cookie jar? What kind of cookie was it?}
      I don't think so? Certainly not in any professional capacity. Some sort of personal cookie? Maybe but I think honesty is one of her real big ethical pillars.
    \item \textit{What subject / topic do they know a lot about that’s completely useless to the direct plot?}
      Lots of Hyrule animal facts. Absolutely enchanted with the natural world. 
    \item \textit{How would they respond to being fired by a good boss?}
      This might be the first thing in her life that would get her to cry. She's determined enough that she'd bounce back a few days later, but I think there'd also be a lot of hanging out under the Zora's River bridge and feeling low first.
    \item \textit{What’s the worst gift they ever received? How did they respond?}
      Once, not long after the events of \textsc{Nintendo, LBW}, Link brought her monster guts from a unique kind of slarok. Something in them began to rot and cause them to bloat not long after he dropped them off and they quickly became so foul in smell she had to magically banish them. Link, with no way of knowing how badly the first set had gone over, came back a few days later with more and she had to send him off to dispose of them. Link felt guilty, but she took it in stride as just a hazard of being a witch, and after that Link made sure the bottles he used to carry monster parts were preservation-quality. 
    \item \textit{What do they tell people they want? What do they actually want?}
      Discussed already in question 1.\footnote{\textit{Supra}, at \pageref{irene:characterqs:lies}.}
    \item \textit{How do they respond when someone doesn’t believe them?}
      Tries to prove it, whatever ``it'' is, probably well past the point of reasonableness. She'll drop her plans for the day if she thinks she can do something to demonstrate she's telling the truth.
    \item \textit{When they make a mistake and feel bad, does the guilt differ when it’s personal versus when it’s professional?}
      No, her conception of being a witch is so brigaded with ethics the personal-professional line is not easiy demarcated.
    \item \textit{When do they feel the most guilt? How do they respond to it?}
      This is another professional one. There's a lot to learn about being a witch and in your first years of indepdent practice you're going to give bad advice a couple times, and those times eat her up. She handles it by doubling down and studying and practicing even harder in following weeks, and probably burns out at some point and has to go back to a steadier pace. If Link is around he'll notice and intervene. I think her Gram probably notices but is less able to persuade Irene to slow in this particular scenario.
    \item \textit{If they committed one petty crime / misdemeanor, what would it be? Why?}
      By process of elimination, vandalism. Honesty is a big deal for Irene, and I don't think she'd steal, and I don't think she'd start a direct fight so brawling is out. But I do think it'd be theoretically possible to make her mad enough she hexed someone's house to smell awful. 
    \item \textit{How do they greet someone they dislike / hate?}
      ``Hello'' and a lazily half-raised hand for a wave.
    \item \textit{How do they greet someone they like / love?}
      One big shout of ``Hey!'' no hand gestures because she's almost certainly at work on something and doesn't need to interrupt herself for people with whom she's close.
    \item \textit{What is the smallest, morally questionable choice they’ve made?}
      She's totally practiced fortune-telling without a formal apprenticeship because it's too interesting not to try. Only, I don't think she thought to tell the people getting their fortunes read that she wasn't formally trained.
    \item \textit{Who do they keep in their life for professional gain? Is it for malicious intent?}
      Irene keeps in close touch with both the Hylian and Lorulian fortune tellers, but has no actual personal connection. Nothing against them, either, they're just\ldots colleagues. 
    \item \textit{What’s a secret they haven’t told serious romantic partners and don’t plan to tell?}\label{irene:characterqs:emotionalcork}
      She's starting the story with zero serious romantic partners to check, but I think once she gets into a serious romantic relationship she really doesn't have a filter? I think this is a product of growing up so isolated that once she has someone other than Gram to talk to on the regular she spills everything.
    \item \textit{What hobby are they good at in private, but bad at in front of others? Why?}
      Birding and wildlife watching. Solo she's able to keep still, quiet and patient. If a friend has come along she's too chatty not to alert animals to her presence.
    \item \textit{Would they rather be invited to an event to feel included or be excluded from an event if they were not genuinely wanted there?}
      Included. Like with \ref{irene:characterqs:emotionalcork} above, because she grew up so isolated I think there's a latent craving for a bit more of a social scene than she has.
    \item \textit{How do they respond to a loose handshake? What goes through their head?}
      I don't think Hyrule has handshakes. But she recovers from social awkwardnesses pretty quickly either way and just doesn't think much on it.
    \item \textit{What phrases, pronunciations, or mannerisms did they pick up from someone / somewhere else?}
    
    \item \textit{If invited to a TED Talk, what topic would they present on? What would the title of their presentation be?}
      
    \item \textit{What do they commonly misinterpret because of their own upbringing / environment / biases? How do they respond when realizing the misunderstanding?}
      As someone who grew up relatively detached from the class system in Hyrule, I think Irene is probably a bit blind to detecting wealth or poverty except in very obvious cases. On the wealthy end, I think certain Old Families in Hyrule have that style where ``[their] tasteful avoidance of ostentation verges on the garish,''\footnote{\textsc{Michael Frayn, Headlong} 21 (1999). Or think of that thing where you can spot old Boston Brahmin families on the street-- allegedly, this is an \emph{old} stereotype and I'm no Bostonian -- by the fact that they're driving what would have been a very fancy car fifteen years ago, but now it is better described as well-preserved, and (again, allegedly) they'll keep driving it for ages yet  because ``it still works'' and to get a new one would be wasteful.} Their house looks simple and unadorned, but has room after room after room deftly concealed by architecture. They're loaded with rupees and have hired help, but they go to the market for themselves once a week so they can't be said to have everything done for them. Those families aren't fooling anyone in Kakariko, but I'm reasonably certain Irene wouldn't glean that Sahasrala was extremely well-off. And then conversely, unless people are literally homeless I'm not sure how easily Irene can clock that their house or belongings signal poverty. There's no reason to revise your opinion of someone on finding out their poor, of course, so I think how she reacts has to be a lot more context-depdendent.\footnote{For example, I think she probably does charge based on a client's ability to pay, and if she knew a client were very rich might charge one hundred rupees for a consult where she might only charge five if she knew they were poor, and probably provide many more consults on the house in the latter case.}
    \item \textit{What language would be easiest for them to learn? Why?}
      This opens up a whole can of worms about the languages of Hyrule I don't want to touch right now. I'll come back to this after more world building.
    \item \textit{What’s something unimportant / frivolous that they hate passionately?}
      Bottles that don't have necks before the cork opening. You're either requiring a witch to go and buy a ton more cork than she really needs, or shrinking the bottle unnecessarily to keep it in line with a small mouth. Either way it's \emph{wasteful}.
    \item \textit{Are they a listener or a talker? If they’re a listener, what makes them talk? If they’re a talker, what makes them listen?}
      Much more a talker, but when people say they have a problem, like some lawyers or doctors, she can't help but shift into listening and start trying to see if there's a witch-y fix for the issue, even if they're just wanting to vent or she wouldn't want to actually counsel them for whatever reason. 
    \item \textit{Who have they forgotten about that remembers them very well?}
      
    \item \textit{Who would they say ‘yes’ to if invited to do something they abhorred / strongly didn’t want to do?}
      Link and Gram. 
    \item \textit{Would they eat something they find gross to be polite?}
      A little gross? Sure. Very gross? No.
    \item \textit{What belief / moral / personality trait do they stand by that you (mun) personally don’t agree with?}
      Kid has her career and her personal identity \emph{way} too closely intermingled and I honestly don't think she shakes it by the end of the story.
    \item \textit{What’s a phrase they say a lot?}
      Zero clue. Just going to have to write this dialogue out and find out the hard way.
    \item \textit{Do they act on their immediate emotions, or do they wait for the facts before acting?}
      She's very much a wait and count to thirty sort of person, and sometimes she'll notice she has a question about a situation she wants answered first, but most the time after she gets to thirty she acts.
    \item \textit{Who would / do they believe without question?}
      Link and Gram.
    \item \textit{What’s their instinct in a fight / flight / freeze / fawn situation?}
      Big scary physical confrontontation? Flight. Any other kind of confrontation (including \emph{moderately}-sized physical confrontations)? Fight.
    \item \textit{What’s something they’re expected to enjoy based on their hobbies / profession that they actually dislike / hate?}\label{irene:characterq:workhates} 
      Not something she's expected to like, but certainly not to mind: Gutting and cleaning monsters. Likewise, handling the guts and tails while they're still wet. She'll do it when it's necessary but there are often times when the monster is especially gross that she'll wonder if she shouldn't switch to being a full time-fortune teller.
    \item \textit{If they’re scared, who do they want comfort from? Does this answer change depending on the type of fear?}
      If it's validation -- you did a good job, you did the right thing -- she'd probably prefer to hear it from Gram. If it's other comfort -- especially reassurance that it's going to be ok, or someone to share feeling sad with -- she'd probably prefer it from Link.
    \item \textit{What’s a simple daily activity / motion that they mess up often?}
      Tying knots. She can tie them, but they just don't \emph{stay}. Most of her stuff that should be tied is magic'd together, which is more energy but so much more reliable.    
    \item \textit{How many hobbies have they attempted to have over their lifetime? Is there a common theme?}
      At least three, but maybe more. Hydrics and water engineering from Zoras. Bug collecting in the woods when she was a child. Fortune telling as a teen and adult, especially palmistry. Ultimatley I think she's fascinated by ways to understand the natural world.
  \end{enumerate}