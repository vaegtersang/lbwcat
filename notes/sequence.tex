%SUPER OLD, figure out what to do with this.

%Note to future self
%If this is a roleswap story, here are the things you mostly want characters to gain
%1) Irene gets to be machiavellian in a sympathetic sort of way, and gains some sympathy for Zelda's POV, which she had initially hated
%2) Link gets to learn the value of practice and hard work through witchcraft and Gram's advice, since he's no longer the hero (and hence no longer automatically good at stuff).
%3) Zelda's time playing the Hero doesn't necessarily change her opinion on the machiavellian stuff, but conversation with Link and experience of how much of it is the same now underscores that, previously, a big asset of Link was that he got information from outside the royal bubble, but using him as official representative and knight subjects him to the same Pauline Keale stuff she's subject to. There's value in him being official, but it's a very steep trade-off.   

\documentclass{article}
\begin{document}
\section{The Story}
\subsection{First}
\begin{enumerate}
\item The order of things: the cycle of threat and revitalizing hero and prosperous governance blah blah \emph{blah}, is revealed to Irene when she discovers she is a sage. A great deal more of history she knows already. 

Irene needn't ask the future. She has, presumably, had it beamed directly into her head, and Hyrule is a land so choked over in mystery and lore that it's impossible for her not to know how the future will remember her, as an eminent footnote in the same hashed and rehashed legend about other people. 
\item So she's getting a second opinion. On the defining myth of her world? Of whether Link really is that Link? On... fuck if she knows, but something can't be right. It can't be as it looks because that would be too awful. 

And frankly too stupid, saving Hyrule from a mystical threat is one thing, but patrolling against Labrynnan bandits -- ordinary bandits -- is another, and (good goddesses!) acting as diplomat is a third still and none of them have anything to do with what Link is good at. 
%(And what if the Peter Principle is written into the universe?) 
%[That's beyond the scope of this story, but.....hmmmmm]
\end{enumerate}
\subsection{Reactions}

\begin{description}
\item[Fortune Teller] Given that Irene isn't really asking for information, what can the fortune teller say?  Not much, but he's the most talented etc etc etc so he probably tries and fails to be a priest and say something ambiguously useful. This goes down in flames. Irene leaves, pays, and flies for a long, long time.
% But even if his response is useless, what does she even ask. This I think is the one where it's most spread out and inchoate: She doesn't want to be fated to be a sidekick, she doesn't want Link to be stuck at a job he neither enjoys nor makes use of his talents (and sends him far away), she doesn't \emph{want}\ldots

\item[HRH Zelda]
\item[DIY]
\end{description}

 She flies and flies and all there is beneath are more signs of this world's claustrophobic past. Towns with borders that stop dead at the sight of woods and bramble. Footpaths overgrown and, even with the canyons gone and curses lifted, without the stamp of a single footprint.
\end{document}