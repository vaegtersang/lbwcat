\documentclass[../FGP.tex]{subfiles}
\renewcommand\thepage{\arabic{page}}
\begin{document}
\section{Lorule}\begin{fragment}\label{nofortunes}


\subsubsection{Draft \thefragment.2}
\ldots The same knowing smile she had seen on the faces of a thousand people who knew \emph{nothing}. 
\end{fragment}

\begin{fragment}In fortunetelling, the ability to divine what information was relevant to the customer was as vital as the actual ability to divine the future was superfluous. The telling was what sold, not the fortunes. The Great Periander, only-and-possibly- foremost fortuneteller in all of Lorule had gone far on this maxim, but was at present finding it deeply challenged , and if the present case was a trying one, it was all the more reason to cling to first principles.\end{fragment}
                    
\begin{fragment}The quicker a fortune was told, the better. The typical customer came, not for a numinous encounter with forces beyond their ken, but to have the information they wanted and be on their way.  The actual telling was, at best, an inconvenience, and at worst the source of terrible anxiety and acute embarrassment. What were, to him, the essential tools of his craft -- his robes and hat, the yellow tablecloth and faintly pink crystal ball, were to customers malcolored and garish eyesores. At this advanced stage of his career he knew customers did not want an explanation of how, absent the aid of these accoutrements, the grueling difficulty of coaxing answers from the great beyond would simply render his trade impossible. What they wanted, dearly, was close their eyes and pretend that they did not see them.\end{fragment} 

\begin{fragment}In years past this reluctance had taken an even darker turn, with efforts to predict the future running afoul of the [Masked monster worshipping folks] taboo of acknowledging that the future existed at all. \end{fragment}

\begin{fragment}This made the look of careful appraisal that the young woman gave to all of his instruments entirely unexpected. She had run her finger across the top of one of his bookshelves and frowned disapprovingly when it came up without dust when, \end{fragment}

\begin{fragment}
``It's not about that. I--'' the peel of a bell cuts her off and her attention is gone from the conversation. She instinctively refuses to give the old fraud the satisfaction of seeing her fret, even as the familiar, if only momentary, panic sets in. It doesn't seem to matter, and the old bat glows anyway. She keeps eye contact and does everything she can to ignore the smug, knowing grin of the smug, know-nothing parody of her Gram as she excuses herself.   

She is on her broom and soaring as soon as she is out of the tent. 
%Important point, she's scared af for L. 
\end{fragment}
\end{document}