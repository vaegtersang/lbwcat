\documentclass[../FGP.tex]{subfiles}
\renewcommand\thepage{\arabic{page}}
\begin{document}
\section{Lorule}\begin{fragment}\label{nofortunes}Self-respecting witches did not meddle, her Gram had drilled this into Irene at the earliest age possible. Self-respecting witches did not bless infants with good character or with the ability to talk to animals. Self-respecting witches did not guarantee good harvests, nor did they guarantee anything. Self-respecting witches especially did not foretell people's futures; and if a self-respecting witch's daily reading of chuu jelly \emph{did} happen to mention some idiot thane destined to become king, she kept it to herself. Irene intended to break none of these rules, now or ever, but also knew better than to hope that Gram, if she ever learned of it, would see the nuances of the present situation.

She likewise was under no illusions that, when Royal duty next found him and Link next vanished for weeks into the monster-infested wilds and returned, as he always did, stuffed impossibly and insufferably full of gossip, word of her visit would somehow not be among the scores of rumors he gathered like rupees. Nor did she have any question about the sense in which the rumors would run; a young woman asking after the fortune of a man her age was even more certainly doomed to romantic misinterpretation than a princess paying honor to her most favored knight. It had ever been thus, but now, on the cusp of what passed for marriageable age in this neck of the woods, it was harder to endure. 

In light of the stubborn facts, Irene had taken no time at all to decide that there would be no rumors, nor a stern talk from her Gram about pride in her craft, and if it required traveling to a fortune-teller a world and a half away, well this was hardly any distance at all to a good broomstick.

It remained a great deal of distance to a young witch.  

\subsubsection{Draft \thefragment.1} In Link and Zelda's telling, Lorule was a land precisely unlike Hyrule in a thousand small ways which added up to a deep, untraceable family resemblance. This description was true, but only in the same strict and literal sense that, with enough great grandparents, she shared a family resemblance with pond moss. Lorule was not some distant cousin of her home, or the reverse to Hyrule's obverse, it was an imposter, and bore no relation from being different than cattails did to hogsheads. She marveled that neither Link nor Zelda had noticed this. 

The Kingdom of Lorule disguised itself poorly and, beyond the copy-cat placed rivers and mountains, was completely dissimilar to the tranquil and prosperous Kingdom she knew. Here was a land pockmarked with hermit-crab towns. A kingdom whose villages' borders stopped dead at the sight of woods and bramble. Whose rivers were left unbridged and footpaths left to overgrowth. Highways that went, even with the canyons gone and curses lifted, without the stamp of a single footprint. Even from the air she felt claustrophobic. Landing, in front of a tiny yellow tent in a grove of trees warped over themselves was worse.

\subsubsection{Draft \thefragment.2}\ldots Towns with borders that stopped dead at the sight of woods and bramble. Rivers left unbridged and footpaths left to overgrowth. Highways that went, even with the canyons gone and curses lifted, without the stamp of a single footprint. 

\subsubsection{Draft \thefragment.3}
\ldots The same knowing smile she had seen on the faces of a thousand people who knew \emph{nothing}. 
\end{fragment}

\begin{fragment}In fortunetelling, the ability to divine what information was relevant to the customer was as vital as the actual ability to divine the future was superfluous. The telling was what sold, not the fortunes. The Great Periander, only-and-possibly- foremost fortuneteller in all of Lorule had gone far on this maxim, but was at present finding it deeply challenged , and if the present case was a trying one, it was all the more reason to cling to first principles.\end{fragment}
                    
\begin{fragment}The quicker a fortune was told, the better. The typical customer came, not for a numinous encounter with forces beyond their ken, but to have the information they wanted and be on their way.  The actual telling was, at best, an inconvenience, and at worst the source of terrible anxiety and acute embarrassment. What were, to him, the essential tools of his craft -- his robes and hat, the yellow tablecloth and faintly pink crystal ball, were to customers malcolored and garish eyesores. At this advanced stage of his career he knew customers did not want an explanation of how, absent the aid of these accoutrements, the grueling difficulty of coaxing answers from the great beyond would simply render his trade impossible. What they wanted, dearly, was close their eyes and pretend that they did not see them.\end{fragment} 

\begin{fragment}In years past this reluctance had taken an even darker turn, with efforts to predict the future running afoul of the [Masked monster worshipping folks] taboo of acknowledging that the future existed at all. \end{fragment}

\begin{fragment}This made the look of careful appraisal that the young woman gave to all of his instruments entirely unexpected. She had run her finger across the top of one of his bookshelves and frowned disapprovingly when it came up without dust when, \end{fragment}

\begin{fragment}
``It's not about that. I--'' the peel of a bell cuts her off and her attention is gone from the conversation. She instinctively refuses to give the old fraud the satisfaction of seeing her fret, even as the familiar, if only momentary, panic sets in. It doesn't seem to matter, and the old bat glows anyway. She keeps eye contact and does everything she can to ignore the smug, knowing grin of the smug, know-nothing parody of her Gram as she excuses herself.   

She is on her broom and soaring as soon as she is out of the tent. 
%Important point, she's scared af for L. 
\end{fragment}
\end{document}